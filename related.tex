\section{Related work}
\label{sec:related}

The idea of applying formal methods to verify \OCaml programs is not new.
Generally speaking, there are mainly two ways:

\paragraph{Semi-automated verification.}

The verified program is annotated by the user to guide the verification tool: preconditions, postconditions, invariants, \etc.
Given this input, the tool generates proof obligations that are mostly automatically discharged.
One may further distinguish two types of semi-automated systems: \emph{foundational} and \emph{non-foundational}.

In \emph{non-foundational} automated verification, the tool and the external solvers it may rely on are part of the trusted computing base.
It is the most common approach and has been widely applied in the literature~\cite{DBLP:journals/jfp/SwamyCFSBY13, DBLP:series/natosec/0001SS17, DBLP:conf/nfm/JacobsSPVPP11, DBLP:conf/icfem/DenisJM22, DBLP:conf/nfm/AstrauskasBFGMM22, DBLP:conf/esop/FilliatreP13}, including to \OCaml by \Cameleer~\cite{DBLP:conf/cav/PereiraR20}, which uses the \Gospel specification language~\cite{DBLP:conf/fm/ChargueraudFLP19} and \WhyThree~\cite{DBLP:conf/esop/FilliatreP13}.

In \emph{foundational} automated verification, the proofs are checked by a proof assistant like \Coq, meaning the automation does not have to be trusted.
To our knowledge, it has been applied to \C~\cite{DBLP:conf/pldi/SammlerLKMD021} and \Rust~\cite{DBLP:journals/pacmpl/GaherSJKD24}.

\paragraph{Non-automated verification.}

The verified program is translated, manually or in an automated way, into a representation living inside a proof assistant.
The user has to write specifications and prove them.

The representation may be primitive, like Gallina for \Coq.
For pure programs, this is rather straightforward, \eg in \texttt{hs-to-coq}\cite{DBLP:conf/cpp/Spector-Zabusky18}.
For imperative programs, this is more challenging.
One solution is to use a monad, \eg in \texttt{coq-of-ocaml}~\cite{coq-of-ocaml}, but it does not support concurrency.

The representation may be embedded, meaning the semantics of the language is formalized in the proof assistant.
This is the path taken by some recent works~\cite{DBLP:books/hal/Chargueraud23, DBLP:journals/pacmpl/GondelmanHPTB23, DBLP:conf/sosp/ChajedTKZ19} harnessing the power of separation logic, in particular the \Iris~\cite{DBLP:journals/jfp/JungKJBBD18} concurrent separation logic.
\Iris is a very important work for the verification of concurrent algorithms.
It allows for a rich, customizable ghost state that makes it possible to design complex \emph{concurrent protocols}.
In our experience, for the lockfree algorithms we considered, there is simply no alternative.

The tool closest to our needs so far is \CFML~\cite{DBLP:books/hal/Chargueraud23}, which targets \OCaml.
However, \CFML does not support concurrency and is not based on \Iris.
The \Osiris~\cite{osiris} framework, still under development, also targets \OCaml and is based on \Iris.
However, it does not support concurrency and it is arguably non-trivial to introduce it since the semantics uses interaction trees~\cite{DBLP:journals/pacmpl/XiaZHHMPZ20}---the question of how to handle concurrency in this context is a research subject.
Furthermore, \Osiris is not usable yet; its ambition to support a large fragment of \OCaml makes it a challenge.
